\documentclass[12pt,a4paper,titlepage,parskip]{scrartcl}
\usepackage[utf8]{inputenc}
\usepackage{amsmath}
\usepackage[german]{babel}
\usepackage{amsfonts}
\usepackage{amssymb}
\usepackage{graphicx}
\author{David Nelles}
\title{Kochbuch}
\date{Stand: \\ \today}

\begin{document}
	\maketitle
	\section{Einleitung}
	Sehr verehrte Leser und Leserinnen, ich freue mich, dass Sie sich mein Kochbuch angeschafft haben. Ich bin ein dualer Student, arbeite in Göttingen beim Deutschen Zentrum für Luft und Raumfahrt und studiere in Mannheim an der Dualen Hochschule Baden Württemberg Informationstechnik. Eigentlich bin ich weder Autor noch ein besonders toller Koch. Trotzdem habe ich mich dazu entschieden ein Kochbuch zu schreiben. Sie fragen sich bestimmt warum: Da ich zu faul bin, mir jeden Abend ein Rezept im Internet oder einem Kochbuch herauszusuchen, koche ich meist frei nach Gefühl. Damit Gerichte, die mir recht gut gelungen sind, nicht verloren gehen schreibe ich diese hier nieder. Natürlich ist das ganze mehr Arbeit als sich ein Gericht aus dem Internet herauszusuchen. Allerdings schreibe ich bei weitem nicht so oft an diesem Kochbuch, wie ich koche, was den Aufwand dieses Buch zu schreiben wiederum drastisch sinkt.
	
	Ich wünsche Ihnen viel Spaß mit den Gerichten und möchte Sie dazu ermutigen, selber mit den Zutaten, den angegebenen Zeitangaben und anderen scheinbar vorgeschriebenen Werten zu experimentieren.
	
	Wenn Sie eine Verbesserung eines Gerichts herausfinden, helfen Sie doch beim verbessern dieser Speise. Dazu müssen sie bloß mit LaTeX und GitHub umgehen können. In GitHub finden Sie dieses Dokument unter der Adresse
\end{document}